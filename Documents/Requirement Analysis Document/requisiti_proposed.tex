\begin{list}{-}{}
	\item{I seguenti requisiti funzionali, appartenenti al Current Envirorment, non sono stati modificati:}	
\end{list}
\begin{quote}	
	\begin{description}		
		\item[RF 1:]				
		\begin{list}{}{}
			\begin{quote}
				\item{RF 1.1}
			\end{quote}					
		\end{list}
		
		\item[RF 2:]				
		\begin{list}{}{}
			\begin{quote}
				\item{RF 2.1}
				\item{RF 2.2}		
			\end{quote}					
		\end{list}
		
		\item[RF 3:]				
		\begin{list}{}{}
			\begin{quote}
				\item{RF 3.1}
				\item{RF 3.2}		
			\end{quote}	
		\end{list}
		\item[RF 4:]	
		\begin{list}{}{}
			\begin{quote}
				\item{RF 4.1}
				\item{RF 4.2}		
			\end{quote}	
		\end{list}
		\item[RF 5]	
		\begin{list}{}{}
		\end{list}
		\item[RF 6]	
		\begin{list}{}{}
		\end{list}
		
	
	\end{description}
\end{quote}

\begin{list}{-}{}
	\item \textbf{RF 2 - Correzione} 
	\begin{quote}Il sistema consente la creazione di una possibile soluzione ai quattro principalei tipi di code smell.\end{quote}
	\begin{quote}
		\begin{description}    		 	
			\item [RF 2.3:]Il sistema fornisce una possibile soluzione al code smell di tipo Blob. 
			
				\begin{description}
					\item[RF 2.3.1:] Il sistema permette di modificare la possibile soluzione al Blob attraverso l'operazione che consente lo spostamento dei metodi applicata alle due classi proposte nell'operazione di "Extract class".		
				\end{description}
			
			\item[RF 2.4:] Il sistema fornisce una possibile soluzione al code smell di tipo Promiscuous Package.  
			
				\begin{description}
					\item[RF 2.4.1:] Il sistema permette di modificare la possibile soluzione al Promiscuous Package attraverso l'operazione di spostamento delle classi applicata ai package proposti nell'operazione di "Extract package".	
				\end{description}
			
			\item[RF 2.5:] Il sistema permette di segnalare come falso positivo una componente indicata come difettosa dal correttore automatico.
			
		\end{description}
	\end{quote}
	\item \textbf{RF 3 - Refactoring:} 
	\begin{quote}Il sistema consente la creazione di una possibile soluzione ai quattro principali tipi di code smell.\end{quote}
	\begin{quote}
		\begin{description}
			\item[RF 3.3:]Il sistema permette allo sviluppatore di risolvere il code smell di tipo Blob.
			
				\begin{description}
					\item[RF 3.3.1:]Il sistema permette di risolvere il code smell di tipo Blob attraverso l'uso dell'operazione di "Extract class".
					\newline		
				\end{description}
					
			\item[RF 3.4:]Il sistema permette allo sviluppatore di risolvere il code smell di tipo Promiscuous Package.
			
				\begin{description}
					\item[RF 3.4.1:]Il sistema permette di risolvere il code smell di tipo Promiscuous Package attraverso l'uso dell'operazione di "Extract package".
					\newline		
				\end{description}
			
		\end{description}
	\end{quote}
	\item \textbf{RF 7 - To Do:} 
	\begin{quote}
	\begin{description}
		\item[RF 7.1:]Il sistema consente, all'avvio del programma, di mostrare all'utente un promemoria per un'eventuale correzione manuale.
	\end{description}
	
		\begin{description}
			\item [RF 7.2: ]Il sistema consente di aggiungere un nuovo promemoria.
		\end{description}	
	\end{quote}
	
	\item	\textbf{RF 8 -  Statistiche di accoppiamento:}
	\newline Il sistema consente di visualizzare le statistiche riguardanti classi e package precedentemente analizzati.
	
	\item	\textbf{RF 9 - Numero modifiche e Revisioni:}
	\newline Il sistema permette di tener traccia del numero di volte che è stato modificato un componente.
	
	\item	\textbf{RF 10 - Analisi rischi:}
	\newline Il sistema consente di visualizzare una stima di pericolosità che potrebbe occorrere nel caso in cui si decide di fare un Refactor.
\end{list}		

\subsection{Requisiti Non Funzionali}
\begin{list}{-}{}
	
	\item \textbf{RNF 1 - Performance}
	\newline Il sistema deve garantire brevi tempi di risposta, in particolare nelle operazioni di analisi del codice.  
	\item \textbf{RNF 2 - Robustezza}
	\newline Il sistema è in grado di funzionare correttamente anche in situazioni anomale o in caso di uso scorretto, notificando l'utente della situazione erronea rilevata, ma senza terminare la propria esecuzione. 
	\item \textbf{RNF 3 - Sicurezza}
	\newline Il sistema si presenta sufficientemente sicuro in quanto il suo ambiente d'azione è locale, quindi non vi è la necessità di protezione da minacce esterne.
	\item \textbf{RNF 4 - Usabilità}
	\newline Il sistema risulta essere di facile utilizzo. L'interfaccia grafica risulta essere intuitiva, agevolando il lavoro dello sviluppatore. 
	\item \textbf{RNF 5 - Manutenibilità}
	\newline Il sistema presenta un elevato grado di manutenibilità, in particolare, favorisce l'aggiunta di nuove funzionalità.
	\item \textbf{RNF 6 - Implementazione}
	\newline Il sistema è realizzato interamente in linguaggio Java, sia parte back-end che front-end. 
	\item \textbf{RNF 7 - Affidabilità}
	\newline Il sistema assicura un'alta affidabilità, riducendo al minimo i casi di arresto anomalo del sistema.
\end{list}

