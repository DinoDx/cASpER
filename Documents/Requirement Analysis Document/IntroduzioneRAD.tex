	   
	   	\subsection{Scope of the system}
	   		Durante il ciclo di vita di un software i cambiamenti sono inevitabili.\\
	   		La manutenzione, che sia per correggere bug o per aggiungere nuove funzionalità, porta a un graduale deperimento del codice, il quale non inficia la correttezza del programma ma porta a debolezze di progettazione: i cosiddetti code smell.\\
	   		Anche uno sviluppatore navigato e ben attento a queste problematiche può cadere in errore, a causa di tempistiche strette o codice ripreso da altri developer.\\
	   	    ASCETIC nasce per permettere a utenti, esperti e non, di refactorizzare il codice in maniera comoda e intelligente.\\
	   		Il plug-in, sviluppato per l'IDE intelliJ IDEA, consente di analizzare il progetto, rilevando 4 possibili tipologie di smell(Blob, Promiscuous Package, Feature Envy, Missplaced Class) e di effettuare un eventuale correzione automatica.
	   		
	   	\subsection{Overview}
	   		ASCETIC (Inizialmente TACOR) è un'opera di reengineering, la quale ha l'obiettivo di migliorare le funzionalità già offerte dal precedente plug-in, offrirne delle nuove ed effettuare un restyling dell'interfaccia grafica per rendere migliore l'esperienza d'uso. ASCETIC si rivolge ad un pubblico composto principalmente da sviluppatori Java che utilizzano l'IDE IntelliJ IDEA e si pone l'obiettivo di facilitare l'aggiustamento del codice sorgente. Lo sviluppatore ha la possibilità di: 
	   		\begin{itemize}
	   			\item Applicare la correzione automatica proposta dal sistema dei Code Smell rilevati durante la fase d'analisi. E' possibile applicare tale soluzione a tutti gli elementi rilevati oppure ad uno specifico insieme selezionato dall'utente.
	   			\item Porre determinati Smell, scelti ad hoc dallo sviluppatore, come "falsi positivi", cosicché il sistema ignori quella determinata sezione di codice fino a nuova modifica. 
	   			\item Utilizzare la funzione "Reminder", la quale fornisce un comodo promemoria così da poter affrontare manualmente la correzione dello smell selezionato. 
	   		
	   		\end{itemize}
	   	
	   	
	   	\subsection{Definitions, acronyms, and abbreviations}
	   	
	   	Code Smell : Sezioni di codice scritte in maniera non ottimale, presentanti debolezze di progettazione che riducono la qualità del codice, non compromettendone il funzionamento.
	    \newline
	   	\\ASCETIC : acronimo per Automated Code Smell Identification and Correction. 