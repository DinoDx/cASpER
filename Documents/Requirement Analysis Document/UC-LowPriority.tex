	\newpage
		\begin{tabular}{|l|p{13cm}|}
			\hline
			\textbf{ID}  & \textbf{UC 2.8} \\ \hline
			\textbf{Nome caso d'uso}  & Statistiche classe/package \\ \hline
			\textbf{Attori partecipanti}  & Sviluppatore \\ \hline
			\textbf{Precondizione}  & UC 1.0 \\ \hline
			\textbf{Flusso di eventi}  & 
			\begin{tabular}{p{6cm}p{6cm}}
				Utente & Sistema  \\
				Dopo l'analisi (UC 1) lo sviluppatore si trova davanti ad una schermata di riepilogo nella quale una volta selezionata una voce tra i risultati proposti, è presente il tasto "Statistiche e Correlazioni" che rimanda ad una nuova finestra.\\
				& il sistema genera una nuova finestra contenente un elenco di \textbf{Associazioni}, \textbf{Aggregazioni}, \textbf{Composizioni} e \textbf{Specializzazioni} che riguardano la classe/package che abbiamo selezionato precedentemente
				\\
				Lo sviluppatore prende visione dei risultati generati dal sistema ed effettua le proprie valutazioni, al termine delle quali chiude semplicemente la finestra e torna alla schermata precedente.
			\end{tabular}\\ \hline
			\textbf{Condizione d'uscita} & L'utente visualizza correttamente le statistiche\\ \hline
			\textbf{Eccezioni}  & \\ \hline
			\textbf{Priorità} & Bassa \\ \hline
			\textbf{Requisiti di qualità}  & \\ \hline 
		\end{tabular}
	
	\newpage
	
	\begin{tabular}{|l|p{13cm}|}
		\hline
		\textbf{ID}  & \textbf{UC 2.9} \\ \hline
		\textbf{Nome caso d'uso}  & Numero di modifiche e revisioni \\ \hline
		\textbf{Attori partecipanti}  & Sviluppatore \\ \hline
		\textbf{Precondizione}  & UC 1.0 \\ \hline
		\textbf{Flusso di eventi}  & 
		\begin{tabular}{p{6cm}p{6cm}}
		 \textbf{Utente} & \textbf{Sistema}  \\
			Dopo l'analisi (UC 1) lo sviluppatore si trova davanti ad una schermata di riepilogo nella quale una volta selezionata una voce tra i risultati proposti, è presente il tasto "Numero di modifiche e revisioni" che rimanda ad una nuova finestra.\\
			& il sistema genera una nuova finestra contenente il numero di modifiche effettuate alla classe/package in esame.
			\\
			Lo sviluppatore prende visione dei risultati generati dal sistema ed effettua le proprie valutazioni, al termine delle quali chiude semplicemente la finestra e torna alla schermata precedente.
		\end{tabular}\\ \hline
		\textbf{Condizione d'uscita}  & L'utente visualizza correttamente le statistiche\\ \hline
		\textbf{Eccezioni}  & \\ \hline
		\textbf{Priorità} & Bassa \\ \hline
		\textbf{Requisiti di qualità}  & \\ \hline 
	\end{tabular}

	\newpage

	\begin{tabular}{|l|p{13cm}|}
		\hline
		\textbf{ID}  & \textbf{UC 2.10} \\ \hline
		\textbf{Nome caso d'uso}  & Analisi Rischi \\ \hline
		\textbf{Attori partecipanti}  & Sviluppatore \\ \hline
		\textbf{Precondizione}  & UC 1.0 \\ \hline
		\textbf{Flusso di eventi}  & 
		\begin{tabular}{p{6cm}p{6cm}}
			Utente & Sistema  \\
			Dopo l'analisi (UC 1) lo sviluppatore si trova davanti ad una schermata di riepilogo nella quale una volta selezionata una voce tra i risultati proposti, è presente il tasto "Stima rischi modifica" che rimanda ad una nuova finestra.\\
			& il sistema genera una nuova finestra contenente una stima di "pericolosità di modifica" basata sulle correlazioni tra classi e il numero di modifiche già effettuate alla classe/package in esame
			\\
			Lo sviluppatore prende visione dei risultati generati dal sistema ed effettua le proprie valutazioni, al termine delle quali chiude semplicemente la finestra e torna alla schermata precedente.
		\end{tabular}\\ \hline
		\textbf{Condizione d'uscita}  & L'utente visualizza correttamente le statistiche \\ \hline
		\textbf{Eccezioni}  & \\ \hline
		\textbf{Priorità} & Bassa \\ \hline
		\textbf{Requisiti di qualità}  & \\ \hline 
	\end{tabular}