	\subsection{Requisiti Funzionali}
		\paragraph{RF 1 - Ricerca code smell}  
		\begin{quote}Il sistema ricerca i code smell all'interno del codice java.	
			\begin{description}
				\item [RF 1.1:]Il sistema permette l'analisi di code smell di tipo Blob,
				\item [RF 1.2:]Il sistema permette l'analisi di code smell di tipo Misplaced Class
				\item [RF 1.3:]Il sistema permette l'analisi di code smell di tipo Feature Envy
				\item [RF 1.4:]Il sistema permette l'analisi di code smell di tipo Promiscuos Package
			\end{description}
		\end{quote}
		
		
		
		\paragraph{RF 2 - Correzione} 
		\begin{quote}Il sistema mostra una possibile soluzione al code smell.
			\begin{description}
				\item [RF 2.1:] Il sistema permette di corregere code smell di tipo Feature Envy.
				\item [RF 2.2:] Il sistema permette di corregere code smell di tipo Misplaced Class.
			\end{description}
		\end{quote}
		
		
		\paragraph{RF 3 - Refactoring} 
		\begin{quote}Il sistema permette allo sviluppatore di risolvere il code smell con degli automatismi del software.
			\begin{description}
				\item [RF 3.1:] Il sistema permette di risolvere code smell di tipo Feature Envy.
				\item [RF 3.2:] Il sistema permette di risolvere code smell di tipo Misplaced Class.
			\end{description}
		\end{quote}
		
		
		\paragraph{RF 4 - Metriche di Qualità} 
		\begin{quote}Il sistema permette di calcolare metriche di qualità.
			\begin{description}
				\item [RF 4.1:] Il sistema permette di calcolare metriche di qualità per i metodi.
				\item [RF 4.2:] Il sistema permette di calcolare metriche di qualità per le classi.
				\item [RF 4.3:] Il sistema permette di calcolare metriche di qualità per i package.
			\end{description}
		\end{quote}
		
		\paragraph{RF 5 - Estrazione dei Topic} 
		\begin{quote}Il sistema consente di estrarre i topic implementati.
		\end{quote}
	
		\paragraph{RF 6 - Verifica Correzione}
		\begin{quote}Il sistema consente di verificare la correttezza del codice.
		\end{quote}  
	
			
	\subsection{Requisiti Non Funzionali}
	
		\begin{list}{-}{}
			
			\item \textbf{RNF 1 - Performance}\newline 
			Il sistema è concepito come utilizzabile da singolo utente, con un'interazione diretta, il tempo di risposta varia in base all'analisi proposta.
			\newline 
			\item \textbf{RNF 2 - Robustezza}
			\newline  Il sistema allo stato attuale non soddisfa il requisito di robustezza.
			\newline
			\item \textbf{RNF 3 - Sicurezza}
			\newline Il sistema si presenta sufficientemente sicuro in quanto il suo ambiente d'azione è locale, quindi non vi è la necessità di protezione da minacce esterne.
			\newline 
			\item \textbf{RNF 4 - Usabilità}
			\newline Il sistema presenta bassi criteri di usabilità, data una scarna e poco intuitiva GUI.
			\newline 
			\item \textbf{RNF 5 - Manutenibilità}\newline 
			Il sistema non presenta alcun tipo di documentazione pregressa, oltre ad adottare scelte implementative poco adatte a tale scopo.
			\newline  
			\item \textbf{RNF 6 - Implementazione}
			\newline Il sistema è realizzato interamente in linguaggio Java, sia parte back-end che front-end.
			\newline 
			\item \textbf{RNF 7 - Affidabilità}
			\newline  Il sistema allo stato attuale non rispetta il requisito di affidabilità.
	  \end{list}